\section{Introduction}

Glioblastoma (GBM) remains a lethal primary brain tumor with a median survival under 18 months~\cite{Stupp2005}. Therapeutic failure is epitomized by multifocal recurrence, where discrete, non-contiguous tumor lesions arise despite gross total resection and chemoradiotherapy. This spatial progression is clinically intractable and often interpreted through the paradigm of stochastic cellular migration, which, while acknowledging diffuse invasion, lacks predictive power for recurrence geography. Classical continuum models (e.g., Fisher-Kolmogorov-Petrovsky-Piskunov equations) describe expanding, wave-like fronts but are intrinsically incapable of spontaneously generating stable, periodic foci from homogeneity---a hallmark of observed multifocal patterns.

\subsection{Related Work and Theoretical Gaps}
Mathematical modeling of glioma has a rich history, dominated by the Proliferation-Invasion (PI) model pioneered by Swanson et al.~\cite{Swanson2000}. This framework describes tumor growth as a reaction-diffusion wave:
\[ \frac{\partial u}{\partial t} = \nabla \cdot (D \nabla u) + \rho u (1 - u/K) \]
While the PI model successfully predicts the diffuse gradient of tumor cells visible on T2-weighted MRI and explains the failure of local resection, it fundamentally solutions are traveling waves. It typically predicts a single, expanding recurrence mass centered on the resection cavity. To explain \textit{multifocality}---distinct, separated lesions---the PI model requires initializing multiple seed points or introducing stochastic jump processes. It cannot explain the \textit{de novo} emergence of satellites from a single mass.

Other approaches have incorporated mechanical stress, metabolic gradients (hypoxia/necrosis), or vascular coupling. However, none have proposed a mechanism for intrinsic spatial symmetry breaking. Our work addresses this gap by importing the Turing instability concept from developmental biology, positing that the tumor's own angiogenic signaling network is sufficient to organize cells into discrete, periodic foci.

In 1952, Alan Turing proposed that periodic structures in morphogenesis---such as leopard spots or digit spacing---emerge from a diffusion-driven instability between a slowly diffusing, self-activating activator and a rapidly diffusing inhibitor~\cite{Turing1952}. This reaction-diffusion framework provides a mathematical basis for pattern formation from an initially uniform state. We hypothesize that the GBM microenvironment supports a biologically analogous Turing instability. The pro-angiogenic, pro-tumorigenic factor VEGF-A, known to be sequestered by heparan sulfate proteoglycans in the extracellular matrix~\cite{Jain2007}, acts as an effectively slow-diffusing activator. Its endogenous inhibitor, soluble VEGFR-1 (sFLT-1), is upregulated by VEGF-A signaling and diffuses freely, satisfying the cross-activation requirement~\cite{Kendall1993}. Critically, the significant disparity in their effective diffusivities, driven by matrix binding, creates the necessary condition for a diffusion-driven instability~\cite{Murray2002}.

Here, we integrate theoretical, computational, and clinical analyses to test the hypothesis that GBM multifocality constitutes an emergent Turing pattern. We derive a biophysically grounded reaction-diffusion model, analytically define its pattern-forming regime, computationally verify nonlinear pattern emergence and its perturbation by therapeutic intervention, and quantitatively validate the model's central spatial prediction against clinical imaging data from a curated cohort of 47 patients. This work reframes recurrence as a predictable, systems-level phenomenon of spatiotemporal self-organization, offering a novel framework for spatial risk prediction and a rational basis for targeting the underlying ecological dynamics.
