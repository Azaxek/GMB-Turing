\appendix
\section{Mathematical Appendix}

\subsection{Linear Stability Analysis Derivation}
\label{app:LSA}

We consider the dimensionless reaction-diffusion system for the morphogens $v$ (activator, VEGF) and $w$ (inhibitor, sFLT-1), assuming the tumor density $u$ evolves on a much slower timescale and can be treated as quasistatic ($u \approx u_0$). The fast subsystem is given by:
\begin{align}
\frac{\partial v}{\partial t} &= \nabla^2 v + \gamma f(v,w), \\
\frac{\partial w}{\partial t} &= d \nabla^2 w + \gamma g(v,w),
\end{align}
where $d = D_w/D_v$ is the diffusivity ratio and $\gamma$ is a scaling factor related to the domain size. The reaction kinetics are:
\begin{align}
f(v,w) &= \alpha_v u_0 - \eta v w - \delta_v v, \\
g(v,w) &= \alpha_w u_0 + \beta_w \frac{v}{1+v} - \eta v w - \delta_w w.
\end{align}
(Note: forms are simplified for analytical tractability here).

The homogeneous steady state $(v_*, w_*)$ is found by solving $f(v_*, w_*) = 0$ and $g(v_*, w_*) = 0$. We linearize around this state by introducing small perturbations:
\begin{align}
v(\mathbf{x}, t) &= v_* + \tilde{v} e^{\lambda t} e^{i \mathbf{k} \cdot \mathbf{x}}, \\
w(\mathbf{x}, t) &= w_* + \tilde{w} e^{\lambda t} e^{i \mathbf{k} \cdot \mathbf{x}}.
\end{align}
Substituting these into the linearized system yields the Jacobian matrix $\mathbf{J}$:
\begin{equation}
\mathbf{J} = \begin{pmatrix} f_v & f_w \\ g_v & g_w \end{pmatrix}_{\!(v_*,w_*)}.
\end{equation}
The characteristic equation for the growth rate $\lambda$ is:
\begin{equation}
\det(\lambda \mathbf{I} - \mathbf{M}) = 0, \quad \text{where } \mathbf{M} = \gamma \mathbf{J} - k^2 \mathbf{D},
\end{equation}
and $\mathbf{D} = \text{diag}(1, d)$. This leads to the dispersion relation:
\begin{equation}
\lambda^2 - \text{Tr}(\mathbf{M})\lambda + \det(\mathbf{M}) = 0.
\end{equation}
For a Turing instability, we require the homogeneous state to be stable to uniform perturbations ($k=0$), which implies:
\begin{align}
\text{Tr}(\mathbf{J}) = f_v + g_w &< 0, \\
\det(\mathbf{J}) = f_v g_w - f_w g_v &> 0.
\end{align}
Additionally, we require instability to spatial perturbations for some $k \neq 0$. This occurs when $\det(\mathbf{M}) < 0$ for some $k$, which leads to the critical condition:
\begin{equation}
d f_v + g_w > 2 \sqrt{d \det(\mathbf{J})}.
\end{equation}
Since $f_v + g_w < 0$, this implies $d \neq 1$. In fact, since we typically need an activator-inhibitor system where $f_v > 0$ and $g_w < 0$, we require $d > 1$ (specifically $d \gg 1$). The critical wavenumber $k_c$ is given by:
\begin{equation}
k_c^2 = \gamma \frac{d f_v + g_w}{2d}, \quad \text{or derived from } \det(\mathbf{M}_{min}) = 0.
\end{equation}

\subsection{Nondimensionalization}
To facilitate numerical stability and analysis, we introduce the following dimensionless variables:
\begin{equation}
\tau = \delta_v t, \quad \mathbf{X} = \sqrt{\frac{\delta_v}{D_v}} \mathbf{x}, \quad V = \frac{v}{v_h}, \quad W = \frac{w}{w_{ref}}.
\end{equation}
Substituting these into the original PDEs eliminates several free parameters and groups the remaining ones into dimensionless groups (e.g., Damk{\"o}hler numbers). The resulting system is solved on the dimensionless domain $[0, L'] \times [0, L']$.

\subsection{Numerical Implementation Details}
\label{app:numerical}
We employ a semi-implicit Crank-Nicolson Alternating Direction Implicit (ADI) method. The time-stepping scheme for the generic reaction-diffusion equation $u_t = D \nabla^2 u + R(u)$ is:
\begin{align}
\frac{u^{n+1/2} - u^n}{\Delta t/2} &= D \delta_x^2 u^{n+1/2} + D \delta_y^2 u^n + R(u^n), \\
\frac{u^{n+1} - u^{n+1/2}}{\Delta t/2} &= D \delta_x^2 u^{n+1/2} + D \delta_y^2 u^{n+1} + R(u^{n+1/2}).
\end{align}
Here, $\delta_x^2$ and $\delta_y^2$ are the standard central difference operators. The nonlinear reaction terms are handled via a predictor-corrector approach using a second-order Runge-Kutta step to maintain $\mathcal{O}(\Delta t^2)$ accuracy.

The anisotropic diffusion for the tumor cells:
\begin{equation}
\nabla \cdot (\mathbf{D}(\mathbf{x}) \nabla u)
\end{equation}
is discretized using a 9-point stencil to account for the cross-derivative terms ($D_{xy} \partial^2 u / \partial x \partial y$) introduced by the tensor rotation. The boundary conditions are zero-flux (Neumann) at the domain edges: $\mathbf{n} \cdot \nabla u = 0$.

\subsection{Parameter Estimation}
\begin{table}[h]
\centering
\begin{tabular}{@{}llp{5cm}@{}}
\toprule
Param & Value & Justification \\ \midrule
$D_v$ & $0.005$ mm$^2$/d & VEGF-A bound to ECM (slow) \\
$D_w$ & $0.20$ mm$^2$/d & sFLT-1 soluble (fast) \\
$K$ & $10^5$ cells/mm$^3$ & Tissue carrying capacity \\
$\rho_0$ & $0.02$ d$^{-1}$ & GBM doubling time $\approx 30$ days \\
$\eta$ & $1.0$ (scaled) & Mass action binding rate \\
\bottomrule
\end{tabular}
\caption{Biophysical parameters used in the simulation.}
\label{tab:params}
\end{table}
