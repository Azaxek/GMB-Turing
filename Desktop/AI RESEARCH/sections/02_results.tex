\section{Results}

\subsection{A biophysical Turing model for glioblastoma patterning}

We formulated a minimal, continuum reaction-diffusion system coupling three key variables: tumor cell density $u(x,t)$ (cells/mm$^3$), VEGF-A concentration $v(x,t)$ (pM), and sFLT-1 concentration $w(x,t)$ (pM). The model incorporates:
\begin{enumerate}
    \item VEGF-enhanced tumor proliferation via a saturating term $\rho_0(1+\phi v/(v_h+v))u(1-u/K)$;
    \item mass-action binding and mutual depletion of VEGF-A and sFLT-1 at rate $\eta$;
    \item constitutive ($\alpha_w u$) and VEGF-inducible ($\beta_w v/(v_s+v)$) production of sFLT-1, establishing the required activator-inhibitor cross-activation;
    \item natural decay of all components.
\end{enumerate}

Diffusion is modeled with a tensor $D_u(x)$ for tumor cells, accounting for anisotropic motility in white matter~\cite{Swanson2000}, and scalar coefficients $D_v$ and $D_w$ for the morphogens. Parameter estimates were derived from biophysical literature: VEGF-A diffusivity is hindered by ECM binding ($D_v \approx 3 \times 10^{-4}$ mm$^2$/day), while sFLT-1 diffuses freely ($D_w \approx 0.012$ mm$^2$/day), yielding a critical dimensionless diffusivity ratio $d = D_w/D_v \approx 40$. Non-dimensionalization produced a simplified system for analysis (Supplementary Note 1).

\begin{figure*}[htbp]
    \centering
    \includegraphics[width=\textwidth]{figures/figure1.png}
    \caption{\textbf{Spontaneous Turing pattern formation in the glioblastoma model.} \textbf{a--d}, Time-series of simulated tumor cell density $u(x,t)$ showing progression from a near-homogeneous state to a stable, periodic spatial pattern. \textbf{e}, The final saturated pattern. \textbf{f}, Azimuthally averaged radial power spectral density (PSD) of the pattern.}
    \label{fig:1}
\end{figure*}

\subsection{Linear stability analysis defines the Turing space}

We performed a linear stability analysis (LSA) around the homogeneous, non-trivial steady state $(u_s, v_s, w_s)$. Treating the slower tumor cell density as a fixed parameter on the faster morphogen timescale, we analyzed the core two-variable $(v,w)$ activator-inhibitor subsystem. The Jacobian $J$ of the reaction kinetics was computed analytically.

The conditions for a Turing instability are twofold: first, stability to uniform perturbations ($\text{Tr}(J) < 0, \det(J) > 0$); second, instability to spatially periodic perturbations induced by diffusion, requiring $d > 1$ and $(D_w J_{11} + D_v J_{22})^2 > 4 D_v D_w \det(J)$. For our baseline parameters, these conditions are robustly satisfied. Solving the dispersion relation $\lambda(k)$ derived from the linearized system reveals a band of unstable wavenumbers $k$. The maximally unstable wavenumber $k_c$, where $\text{Re}(\lambda(k))$ peaks, defines the predicted characteristic wavelength of the emerging pattern: $\Lambda_c = 2\pi/k_c$. Our analysis yields $\Lambda_c = 2.76$ cm, providing an \textit{a priori}, parameter-dependent prediction (Supplementary Note 2).

\subsection{Numerical simulations confirm spontaneous pattern formation}
We solved the full, nonlinear, time-dependent PDE system on large 2D Cartesian domains (typically $10 \times 10$ cm$^2$, $\Delta x \approx 0.02$ cm) using a GPU-accelerated, semi-implicit Crank-Nicolson Alternating Direction Implicit (ADI) scheme. The system progressed through the canonical stages of a Turing instability: initial symmetry breaking, a linear regime of mode selection, and nonlinear saturation into a stable, stationary pattern of discrete tumor cell density peaks (Fig.~\ref{fig:1}).

The simulation begins with a homogeneous background of tumor cells ($u_0 = 0.1K$) perturbed by $1\%$ Gaussian noise. Within the first 50 days, the high-frequency components of the noise decay, leaving only the unstable modes predicted by the LSA (Fig.~\ref{fig:1}, second panel). By $t=150$ days, distinct high-density foci begin to emerge, suppressing tumor growth in their immediate vicinity via the long-range inhibition mechanism. By $t=400$ days, the pattern stabilizes into a static array of spots. Fourier spectral analysis of the final saturated pattern revealed a sharp, dominant peak in the azimuthally averaged power spectral density (PSD). The corresponding wavelength was $\Lambda_{\text{sim}} = 2.84 \pm 0.07$ cm ($n=10$), demonstrating excellent agreement with the linear prediction $\Lambda_c = 2.76$ cm.

\subsection{The ``post-surgical pattern reset'' phenomenon}
To model therapeutic perturbation, we simulated the surgical removal of the dominant tumor peak at pattern saturation ($t=400$ days). This was implemented by setting $u=0$ and $v=0$ within a circular region of radius $R=1.5$ cm centered on the primary peak (Fig.~\ref{fig:2}).

This intervention triggered a dynamic ``pattern reset.'' The abrupt withdrawal of the primary activator source caused a rapid collapse of the long-range sFLT-1 inhibitory gradient. The decay time constant $\tau \sim \max(1/\delta_w, R^2/D_w) \approx 35$ days is governed by inhibitor degradation and diffusion. Consequently, previously suppressed satellite foci were released from inhibition, entering a phase of accelerated, exponential-like growth (Fig.~\ref{fig:2}, third panel). Quantitatively, the effective doubling time of the total tumor burden decreased from 85 days pre-resection to 45 days post-resection (Fig.~\ref{fig:2}e). The volume of the largest satellite lesion increased 3.2-fold within 60 days post-resection, providing a mechanistic explanation for the rapid recurrence often seen clinically.

\begin{figure*}[htbp]
    \centering
    \includegraphics[width=\textwidth]{figures/figure2.png}
    \caption{\textbf{Post-surgical pattern reset and accelerated satellite growth.} \textbf{a--d}, Tumor cell density maps before and after simulated gross total resection. \textbf{e}, Growth kinetics showing accelerated doubling times post-resection.}
    \label{fig:2}
\end{figure*}


\subsection{Anatomical heterogeneity modulates patterning}
Real brains are not isotropic. To assess clinical relevance, we simulated patterning on an anatomically heterogeneous computational domain derived from a synthetic coronal brain slice. The domain incorporated a lateral ventricle (zero-Dirichlet boundary, sink for morphogens) and anisotropic diffusion tensors representing white matter tracts.

\begin{figure}[htbp]
    \centering
    \includegraphics[width=0.8\columnwidth]{figures/figure3.png}
    \caption{\textbf{Pattern modulation by anatomical heterogeneity.} \textbf{a}, Computational domain implementing brain anatomy. \textbf{b}, Final simulated tumor cell density pattern showing elongation along white matter tracts and ventricular avoidance.}
    \label{fig:3}
\end{figure}

The resulting tumor pattern was profoundly modulated by anatomy. Tumor peaks aligned and elongated along the white matter fiber direction ($D_{white} \gg D_{gray}$), recapitulating the ``butterfly glioma'' morphology. Furthermore, peaks were consistently excluded from periventricular regions, mirroring clinical observations of ventricular repulsion (Fig.~\ref{fig:3}). The characteristic Turing wavelength ($\Lambda \approx 2.8$ cm) was approximately preserved within the isotropic parenchymal regions, demonstrating robustness to geometric domain complexity.


\subsection{Clinical imaging validation}
We conducted a rigorous quantitative spatial analysis of a curated cohort from the BraTS 2025-2026 datasets ($n=47$)~\cite{Menze2015,Bakas2017}.

First, the distribution of Euclidean distances from the primary lesion centroid to all satellite centroids was decisively non-Poissonian ($p<0.001$). A Gaussian kernel density estimate revealed a peak at $d_{\text{peak}} = 2.91$ cm (95\% CI: 2.63--3.19 cm) (Fig.~\ref{fig:4}a).

Second, the pair correlation function $g(r)$ for all satellite lesions pooled across patients showed a statistically significant peak at $r=2.87$ cm ($p<0.005$) (Fig.~\ref{fig:4}b). This indicates a characteristic length scale of self-organization that is not explained by random seeding.

Third, patient-specific spectral analysis yielded a median empirical wavelength of $\Lambda_{\text{emp}} = 2.89$ cm (IQR: 2.52--3.31 cm) (Fig.~\ref{fig:4}c). This distribution is statistically indistinguishable from the model-predicted $\Lambda_{\text{sim}} = 2.84$ cm ($p=0.42$).

\begin{figure*}[htbp]
    \centering
    \includegraphics[width=\textwidth]{figures/figure4.png}
    \caption{\textbf{Clinical validation of a characteristic spatial scale.} \textbf{a}, Histogram of primary-satellite distances peaking at 2.91 cm. \textbf{b}, Pair correlation function $g(r)$ showing significant peak at 2.87 cm. \textbf{c}, Patient-specific empirical wavelengths matching model predictions.}
    \label{fig:4}
\end{figure*}
