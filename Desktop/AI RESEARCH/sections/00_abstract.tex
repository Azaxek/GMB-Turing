Glioblastoma recurrence is frequently multifocal, with non-contiguous lesions challenging treatment. Prevailing models view this as stochastic cellular spread. We propose multifocality as a deterministic, emergent pattern driven by a Turing-type reaction-diffusion instability. We identify Vascular Endothelial Growth Factor-A (VEGF-A) and its soluble decoy receptor (sFLT-1) as an effective activator-inhibitor pair. Linear stability analysis of a biophysical model defines a ``Turing space'' requiring a high diffusivity ratio ($d \approx 40$), predicting a pattern wavelength of 2.84 cm. High-fidelity simulations confirm spontaneous pattern emergence and a ``post-surgical pattern reset'' where resection accelerates satellite growth. Quantitative analysis of multifocal glioblastoma MRI (BraTS, $n=47$) reveals a non-Poissonian inter-lesion distance distribution with a characteristic scale of 2.91 cm, matching model predictions. This Turing framework provides a predictive model for recurrence geography, informing targeted radiotherapy.

\textbf{Keywords:} Glioblastoma, Turing Pattern, Reaction-Diffusion, VEGF-A, sFLT-1, Multifocal Recurrence, Spatial Modeling, Radiotherapy Planning
