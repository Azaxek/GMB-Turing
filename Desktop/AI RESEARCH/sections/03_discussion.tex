\section{Discussion}

This study establishes, for the first time to our knowledge, a comprehensive Turing patterning framework for glioblastoma multifocality. By integrating a biophysical activator-inhibitor model with high-fidelity simulations and clinical image analysis, we demonstrate that the spatial distribution of GBM recurrence is not stochastic but deterministic.

\subsection{Biological Implications of Self-Organization}
The core finding---that GBM multifocality arises from a reaction-diffusion instability---fundamentally challenges the prevailing ``seed-and-soil'' or stochastic dissemination models. In those frameworks, satellite lesions are viewed as random metastases. Our model suggests they are inevitable, emergent structures defined by the tumor's own reaction kinetics. The specific identification of the VEGF-A/sFLT-1 axis provides a molecular basis for this. The high diffusivity ratio ($d \approx 40$) required for patterning is naturally achieved by the differential binding affinities of the two molecules: VEGF-A is tightly sequestered by the ECM (low effective $D$), while sFLT-1 is soluble (high $D$). This creates the ``local activation, long-range inhibition'' dynamic essential for Turing spots.

\subsection{Mechanisms of Post-Surgical Recurrence}
Our model offers a novel mechanistic explanation for the ``explosive'' recurrence often seen after Gross Total Resection (GTR). Clinically, this is attributed to the removal of ``bulk'' tumor allowing dormant cells to wake up. Our model refines this: the bulk tumor acts as the primary source of the inhibitor (sFLT-1). Its removal causes a rapid decay of the inhibitory field, while the activator (VEGF-A), being matrix-bound, may persist locally or be rapidly regenerated by microscopic seeds. This sudden imbalance pushes the system from a stable Pattern-1 state (dominated by the primary mass) to a Pattern-2 state (multiple satellites). This suggests that post-operative care should focus not just on killing cells, but on maintaining the inhibitory field---perhaps via intrathecal sFLT-1 delivery.

\subsection{Translational Potential: The `Turing Metric'}
The most significant clinical application is observing that the wavelength $\Lambda$ is a predictable patient-specific biomarker. If we can estimate a patient's diffusion parameters (using DTI) and reaction rates (using proteomic biopsy analysis), we can predict their specific $\Lambda$.
\begin{enumerate}
    \item \textbf{Risk Mapping}: We can generate a probability map of \textit{where} satellites will form, even before they are visible on MRI.
    \item \textbf{Dose Painting}: Radiotherapy plans (RT) currently apply uniform margins. A Turing-informed RT plan would boost dose to the predicted ``nodes'' of the standing wave pattern, potentially effectively sterilizing the sites of future recurrence.
\end{enumerate}

\subsection{Limitations and Future Directions}
The model assumes a continuum of cell density, which breaks down at very low numbers. Future hybrid discrete-continuum models should address the stochastic nucleation of spots. Additionally, we simplified the 3D brain geometry to coronal slices; full 3D simulations on patient-specific meshes are the next computational frontier. Finally, while BraTS data supports the spatial statistics, longitudinal validation in a single cohort from diagnosis to death is necessary to confirm the temporal evolution predicted by our ``pattern reset'' hypothesis.

In conclusion, we present a novel, integrative framework that shifts the conceptualization of GBM progression from stochastic dissemination to constrained self-organization. By decoding the morphogenetic principles of recurrence, this work lays the foundation for a more predictive and spatially precise neuro-oncology.
