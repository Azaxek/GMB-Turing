\textbf{Fig. 1 | Spontaneous Turing pattern formation in the glioblastoma model.} \textbf{a--d}, Time-series of simulated tumor cell density $u(x,t)$ showing progression from a near-homogeneous state to a stable pattern. \textbf{e}, The final saturated pattern. \textbf{f}, Azimuthally averaged radial PSD showing peak at $k_{\text{peak}} \approx 2.21$ cm$^{-1}$.

\textbf{Fig. 2 | Post-surgical pattern reset and accelerated satellite growth.} \textbf{a}, Pre-resection pattern. \textbf{b}, State immediately after resection. \textbf{c}, Accelerated growth of satellite foci at 50 days. \textbf{d}, Pattern reset at 100 days. \textbf{e}, Growth kinetics showing shortened doubling time.

\textbf{Fig. 3 | Pattern modulation by anatomical heterogeneity.} \textbf{a}, Computational domain. \textbf{b}, Final simulated pattern reflecting white matter anisotropy and ventricular boundaries.

\textbf{Fig. 4 | Clinical validation of a characteristic spatial scale.} \textbf{a}, Histogram of primary-satellite distances ($n=142$) peaking at 2.91 cm. \textbf{b}, Pair correlation function $g(r)$ ($p<0.005$). \textbf{c}, Patient-specific empirical wavelengths indistinguishable from model predictions ($p=0.42$).
