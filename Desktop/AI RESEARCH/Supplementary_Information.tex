\documentclass[11pt,a4paper]{article}
\usepackage[utf8]{inputenc}
\usepackage[T1]{fontenc}
\usepackage{amsmath,amssymb}
\usepackage{graphicx}
\usepackage{booktabs}
\usepackage[left=2.5cm,right=2.5cm,top=2.5cm,bottom=2.5cm]{geometry}
\usepackage{hyperref}

\title{\Large\textbf{Supplementary Information} \\[0.5em] \normalsize A Turing Patterning Mechanism Deterministically Predicts the Spatiotemporal Geography of Glioblastoma Multifocal Recurrence}
\author{}
\date{}

\begin{document}
\maketitle

\tableofcontents
\newpage

\section{Supplementary Note 1: Full Nondimensionalization}
We introduce the following characteristic scales:
\begin{align*}
    x^* &= \sqrt{D_v / \delta_v}, & t^* &= 1/\delta_v, \\
    u^* &= K, & v^* &= \alpha_v K / \delta_v, & w^* &= \alpha_w K / \delta_w.
\end{align*}
Substituting dimensionless variables $\tilde{x} = x/x^*$, $\tilde{t} = t/t^*$, $\tilde{u} = u/u^*$, etc., we obtain the dimensionless system:
\begin{align}
    \frac{\partial \tilde{u}}{\partial \tilde{t}} &= \tilde{D}_u \nabla^2 \tilde{u} + \tilde{\rho} (1 + \tilde{\phi} \tilde{v}) \tilde{u} (1 - \tilde{u}) - \tilde{\delta}_u \tilde{u}, \\
    \frac{\partial \tilde{v}}{\partial \tilde{t}} &= \nabla^2 \tilde{v} + \tilde{u} - \tilde{\eta} \tilde{v} \tilde{w} - \tilde{v}, \\
    \frac{\partial \tilde{w}}{\partial \tilde{t}} &= d \nabla^2 \tilde{w} + \tilde{\alpha}_w \tilde{u} + \tilde{\beta}_w \tilde{v} - \tilde{\eta} \tilde{v} \tilde{w} - \tilde{\delta}_w \tilde{w},
\end{align}
where $d = D_w/D_v$ is the critical diffusivity ratio.

\section{Supplementary Note 2: Linear Stability Analysis Derivation}
The Jacobian of the $(v, w)$ subsystem at steady state $(v_s, w_s)$, treating $u = u_s$, is:
\begin{equation}
    \mathbf{J} = \begin{pmatrix} J_{11} & J_{12} \\ J_{21} & J_{22} \end{pmatrix} = \begin{pmatrix} -\eta w_s - \delta_v & -\eta v_s \\ \beta_w / (1+v_s)^2 - \eta w_s & -\eta v_s - \delta_w \end{pmatrix}.
\end{equation}
The dispersion relation is:
\begin{equation}
    \det(\lambda \mathbf{I} - \mathbf{J} + k^2 \mathbf{D}) = 0,
\end{equation}
which expands to:
\begin{equation}
    \lambda^2 - \lambda \left[ \text{Tr}(\mathbf{J}) - k^2(1+d) \right] + \left[ \det(\mathbf{J}) - k^2(J_{11} d + J_{22}) + k^4 d \right] = 0.
\end{equation}
A Turing instability occurs when Re($\lambda$) $> 0$ for some $k \neq 0$, while Re($\lambda$) $< 0$ for $k = 0$. This requires:
\begin{enumerate}
    \item $\text{Tr}(\mathbf{J}) < 0$ (stability without diffusion).
    \item $\det(\mathbf{J}) > 0$ (stability without diffusion).
    \item $d J_{11} + J_{22} > 0$ (requires $d \gg 1$ if $J_{11} > 0$ and $J_{22} < 0$).
    \item $(d J_{11} + J_{22})^2 > 4 d \det(\mathbf{J})$.
\end{enumerate}
The critical wavenumber $k_c$ is found by maximizing Re($\lambda(k)$):
\begin{equation}
    k_c^2 = \frac{d J_{11} + J_{22}}{2d}.
\end{equation}

\section{Supplementary Table 1: Model Parameters}
\begin{table}[h]
\centering
\begin{tabular}{@{}llll@{}}
\toprule
\textbf{Parameter} & \textbf{Symbol} & \textbf{Value} & \textbf{Source} \\ \midrule
Tumor diffusion (grey) & $D_{u,\text{gray}}$ & $0.01$ mm$^2$/d & Swanson et al. (2000) \\
Tumor diffusion (white) & $D_{u,\text{white}}$ & $0.1$ mm$^2$/d & Swanson et al. (2000) \\
VEGF-A diffusion & $D_v$ & $3 \times 10^{-4}$ mm$^2$/d & Jain et al. (2007) \\
sFLT-1 diffusion & $D_w$ & $0.012$ mm$^2$/d & Kendall \& Thomas (1993) \\
Diffusivity ratio & $d = D_w/D_v$ & $\approx 40$ & Derived \\
Proliferation rate & $\rho_0$ & $0.02$ d$^{-1}$ & Harpold et al. (2007) \\
Carrying capacity & $K$ & $10^5$ cells/mm$^3$ & Estimated \\
VEGF-A decay & $\delta_v$ & $0.1$ d$^{-1}$ & Ferrara et al. (2003) \\
sFLT-1 decay & $\delta_w$ & $0.1$ d$^{-1}$ & Estimated \\
Binding rate & $\eta$ & $1.0$ (scaled) & Mass action \\
\bottomrule
\end{tabular}
\caption{Biophysical parameters used in the Turing model.}
\end{table}

\section{Supplementary Table 2: Patient Demographics}
\begin{table}[h]
\centering
\begin{tabular}{@{}ll@{}}
\toprule
\textbf{Characteristic} & \textbf{Value (n=47)} \\ \midrule
Age, median (IQR) & 58 (49--67) years \\
Sex, male (\%) & 28 (60\%) \\
IDH status & All wildtype \\
Number of lesions, median (range) & 4 (3--9) \\
Primary lesion volume, median (IQR) & 32 (18--55) cm$^3$ \\
Satellite lesion volume, median (IQR) & 2.1 (0.8--5.2) cm$^3$ \\
\bottomrule
\end{tabular}
\caption{Demographic and clinical characteristics of the multifocal GBM cohort.}
\end{table}

\section{Supplementary Figure Legends}
\textbf{Extended Data Fig. 1.} Dispersion relation $\lambda(k)$ for baseline parameters, showing the band of unstable wavenumbers.

\textbf{Extended Data Fig. 2.} Global sensitivity analysis. First-order Sobol indices for model outputs (wavelength, saturation time, peak count) across 5,000 Latin Hypercube samples.

\textbf{Extended Data Fig. 3.} 3D visualization of lesion centroids (MNI space) from the BraTS cohort, showing spatial distribution relative to ventricles and corpus callosum.

\end{document}
